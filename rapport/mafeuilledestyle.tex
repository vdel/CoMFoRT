%déclaration des paquets divers et variés
\usepackage[francais]{babel}
\usepackage[utf8x]{inputenc}
\usepackage[T1]{fontenc}
\usepackage{amsmath,amssymb,amsopn} %amsopn pour les nouveaux opérateurs
\usepackage{mathrsfs} %pour la police anglaise \mathscr{}
\usepackage{array} %pour les nouveaux types de colonne à côté des graphiques (\begin{array}{b})
\usepackage{palatino} %police plus sympa
\usepackage{mathpazo} %police plus sympa
\usepackage{calc} %faire des + - * / dans le code
\usepackage[dvips,a4paper]{geometry} %changer les marges simplement
\usepackage{enumerate} %pour les i) ii)
\usepackage{xspace} %pour la fin des commandes \dl, \sev, pour pouvoir faire \dl. et \dl mot après.





%latex mange trop de marges
\geometry{lmargin=20mm,rmargin=20mm,tmargin=20mm,bmargin=23mm}

%gestion de ma nouvelle hiérarchie : partie
\renewcommand{\part}{\partie}
\newcounter{partie}
\newlength{\lligne}
\newlength{\lmot}
\renewcommand{\thepartie}{\Roman{partie}}
\newcommand{\partiename}{Partie}
\newcommand{\partiemark}[1]{\markboth{\thepartie.\quad#1}{\thepartie.\quad#1}}
\newcommand{\partie}[1]{
  \newpage
  \stepcounter{partie}
  \setcounter{section}{0}
  %pour la table des matières
  \addcontentsline{toc}{partie}{\bfseries\scshape\numberline{\thepartie}#1}
  \partiemark{#1}
  %UGLY HACK ! pour avoir des en-têtes corrects malgré tout
  \renewcommand{\rightmark}{\thepartie.\quad#1}
  %calcul de la largeur de la ligne à gauche de ``Partie 1''
  \settowidth{\lmot}{\large\textsc{Partie \thepartie}}
  %le \textwidth/2 est pour le style twocolumn, ne marche plus sans l'option ``twocolumn''
  \setlength{\lligne}{(\columnwidth - \lmot -3em)/2}
  %tracé effectif
  \begin{center}
  \rule[.5ex]{\lligne}{.05em}\quad{\large\textsc{Partie \thepartie}}\quad
  \rule[.5ex]{\lligne}{.05em}
  {\huge\textsc{#1}}
  \hfil\rule[1.5ex]{\lligne + \lligne + 2em + \lmot}{.05em}
  %\vspace{-2ex}
  %il reste un bug : la deuxième ligne est un pixel trop à gauche
  \end{center}
}


%le look du titre
\makeatletter
% Une commande sembleble à \rlap ou \llap, mais centrant son argument
\def\clap#1{\hbox to 0pt{\hss #1\hss}}%
% Une commande centrant son contenu (à utiliser en mode vertical)
\def\ligne#1{%
  \hbox to \hsize{%
    \vbox{\centering #1}}}%
% Une comande qui met son premier argument à gauche, le second au 
% milieu et le dernier à droite, la première ligne ce chacune de ces
% trois boites coïncidant
\def\haut#1#2#3{%
  \hbox to \hsize{%
    \rlap{\vtop{\raggedright #1}}%
    \hss
    \clap{\vtop{\centering #2}}%
    \hss
    \llap{\vtop{\raggedleft #3}}}}%
% Idem, mais cette fois-ci, c'est la dernière ligne
\def\bas#1#2#3{%
  \hbox to \hsize{%
    \rlap{\vbox{\raggedright #1}}%
    \hss
    \clap{\vbox{\centering #2}}%
    \hss
    \llap{\vbox{\raggedleft #3}}}}%
% La commande \maketitle
\def\maketitle{%
  \thispagestyle{empty}\vbox to \vsize{%
    \haut{}{\large \@author}{}
    \vfill
    \begin{flushleft}
      \huge \@title
    \end{flushleft}
    \par
    \hrule height 1pt
    \par
    \begin{flushright}
      \large \@subtitle \par
    \end{flushright}
    \vfill
    \vfill
    \bas{}{\@location, \@date}{}
    }%
  \clearpage
  }
% Les commandes permettant de définir la date, le lieu, etc.
\def\date#1{\def\@date{#1}}
\def\subtitle#1{\def\@subtitle{#1}}
\def\title#1{\def\@title{#1}}
\def\location#1{\def\@location{#1}}
\def\author#1{\def\@author{#1}}
% Valeurs par défaut
\date{\today}
\author{}
\title{}
\location{Paris}
\subtitle{}
\makeatother


%le look des sections
\makeatletter
\renewcommand{\section}{\@startsection{section}{1}{0em}%
  {1.5\baselineskip}{.8\baselineskip}  %marge verticale avant et après le titre
  {\Large\bfseries}} %avant \center\large\bfseries
\makeatother
\renewcommand{\sectionmark}[1]{\markboth{\thesection.\ #1}{\rightmark}}
\renewcommand{\thesection}{\thepartie.\arabic{section}}


%le look des subsections
\makeatletter
\renewcommand{\subsection}{\@startsection{subsection}{2}{1.7em} %avant: 0em
  {1.2\baselineskip}{.4\baselineskip}
  {\bfseries\sffamily}}
\makeatother
\renewcommand{\thesubsection}{\alph{subsection}.\!}


%le look des subsubsections
\makeatletter
\renewcommand{\subsubsection}{\@startsection{subsubsection}{3}{2.4em}
  {.5\baselineskip}{.125\baselineskip}
  {\sffamily}}
\makeatother
\renewcommand{\thesubsubsection}{\Large$\star$}


%pour la table des matières, personnalisation de l'affichage
\makeatletter
\newcommand\l@partie{\vspace{2em}\@dottedtocline{-1}{0em}{2.3em}}
\renewcommand{\@dotsep}{1000}
\renewcommand\l@section{\@dottedtocline{1}{.5em}{3.2em}}
\renewcommand\l@subsection{\@dottedtocline{1}{2em}{1.7em}}
\makeatother


%les caractères jolis de fin de section
\DeclareFontFamily{OT1}{sincos}{}
\DeclareFontShape{OT1}{sincos}{m}{n}{<-> sincos}{}
\newcommand{\sincos}{\fontencoding{OT1}\fontfamily{sincos}\fontseries{m}%
  \fontshape{n}\fontsize{10}{12}\selectfont}


%les en-têtes
\pagestyle{plain}
\makeatletter
\renewcommand{\@evenfoot}{\hfil\itshape - \thepage\ -\hfil}
\renewcommand{\@oddfoot}{\@evenfoot}
\renewcommand{\@evenhead}{\textsc{\bfseries\large\rightmark}\ \hfil\ {\footnotesize\leftmark}}
\renewcommand{\@oddhead}{{\footnotesize\leftmark}\ \hfil\ \textsc{\bfseries\large\rightmark}}
\makeatother

